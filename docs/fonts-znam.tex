\documentclass[11pt]{article}
\usepackage{polyglossia}
   \setmainlanguage{english}
   \setotherlanguage{churchslavonic}
\usepackage{fontspec,luacolor,metalogo}
\setmainfont[Script=Latin,Ligatures=TeX,HyphenChar=-]{Linux Libertine O}
\setmonofont[Scale=MatchLowercase]{FiraSlav}
\newfontfamily\churchslavonicfont[Script=Cyrillic,Ligatures=TeX,Scale=MatchLowercase,HyphenChar=_]{PonomarUnicode.otf}
\newfontfamily\musicFont[Path=../Mezenets/,Extension=.otf,Scale=1.5]{MezenetsUnicode}
\newfontfamily{\oldstyle}[Path=../Mezenets/,Extension=.otf,Scale=1.5,StylisticSet=1]{MezenetsUnicode}
\newfontfamily{\newstyle}[Path=../Mezenets/,Extension=.otf,Scale=1.5,StylisticSet=2]{MezenetsUnicode}
\newfontfamily{\contracted}[Path=../Mezenets/,Extension=.otf,Scale=1.5,StylisticSet=3]{MezenetsUnicode}
\usepackage{churchslavonic}
\usepackage{cu-kruk}
\usepackage{pdfpages}
\usepackage[unicode,
  pdftitle={fonts-znam: fonts for Znamenny Notation},
  pdfauthor={Aleksandr Andreev},
  pdfkeywords={Church Slavonic, Old Church Slavonic, Church Slavic, Znamenny Notation, OpenType, Unicode, церковнославянский, старославянский, знаменная нотация, крюки, столповая нотация}
]{hyperref}

\def\fileversion{0.4}
\let\cuKrukFont=\musicFont
\def\pkg#1{\textsf{#1}}
\def\cs#1{\texttt{\textbackslash #1}}
%

\begin{document}
\title{\pkg{fonts-znam} --- fonts for Znamenny Notation}
\author{Aleksandr Andreev and Nikita Simmons\\version~\fileversion}
\date{\today}
\maketitle

\tableofcontents

\section{Introduction}
This package provides OpenType-CFF PostScript fonts (\texttt{.OTF} fonts)
for typesetting Znamenny Musical Notation encoded in Unicode. Znamenny
Notation has been proposed for inclusion in Unicode by the authors.
The codechart indicating the codepoints used for Znamenny Notation
is presented in the Appendix to this document. The latest version of
the fonts may be downloaded from the \href{https://sci.ponomar.net/music.html}
{Slavonic Computing Initiative website}.

\cuKinovar{Warning:} the codepoints used in this font have been proposed
by the authors for inclusion into the Unicode standard, but have not yet
been approved by the Unicode Technical Committee (UTC). Codepoints are not
final until approved by the UTC, and so the encoding model is subject to change.
As such, these fonts are provided as a BETA-stage software, and should be used
for SOFTWARE-TESTING PURPOSES ONLY.

\cuKinovar{Color Data}: the fonts provide information about the color of some
glyphs (such as the cinnabar marks) via data in the COLR and CPAL tables.
However, COLR / CPAL technology is presently not supported by some software.
Color fonts are supported in \href{https://www.colorfonts.wtf/}{modern browsers
and advanced publishing software} and partially supported in
\href{https://office-watch.com/2019/color-font-support-in-microsoft-office-the-story-so-far/}{Microsoft
Office 365}. Color fonts are not supported in \XeTeX{} or \LuaTeX{}, however
the \pkg{churchslavonic} \LaTeX{} package provides the \texttt{autocolormarks} option,
which uses the \cs{cuKinovar} macro to automatically typeset the cinnabar marks in red.

In other software that does not support color fonts, cinnabar marks will appear in black.
Additionally, attempting to manually color the marks using formatting will generally
break the OpenType glyph positioning rules, leading to improper mark positioning.
You are encouraged to only use the fonts in software that supports color fonts. You
can report your results \href{https://github.com/slavonic/fonts-znam/issues/19}{in
the issue tracker}. The authors plan to add support for OpenType-SVG as well, though
it is not clear that this would improve support in software.

\section{Mezenets Unicode}

Mezenets Unicode is a font intended for typesetting Type~A Znamenny Notation (notation
that includes \emph{priznaki}), though it may also be used for Type~B Notation (without
\emph{priznaki}). It also provide glyphs necessary for Demestvenny notation. An example in
Znamenny Notation is presented in Figure~\ref{mezenets}.

\begin{figure}[tbp]
%\begin{adjustbox}[width=\textwidth]
\begin{churchslavonic}
\noindent
\cuKrukPara[sylRuleHeight=0.02em]{
𜽟 𜾆𜽂𜼱𜼇 𜽗𜼆 𜽐𜼱𜼢 𜽖𜽃𜼅 𜽜𜼄𜼦 𜾖𜼆𜼤 𜽝𜽃𜼅𜼤𜽘 𜾆𜽃𜼱𜼈 𜽞 𜾬𜽃𜼈𜼤 𜽖𜽃𜼅 𜾒𜼰𜼺𜼆𜼣 𜾖𜼆𜼤 𜾋𜽃𜼅 𜾆𜼄 𜽝𜼄 𜽐𜽃𜼅 𜽗𜽂𜼄 𜽞 𜽐𜽃𜼰𜼅 𜽖𜽃𜼢 𜽗𜽃𜼢 𜾆𜾅𜼇 𜽝𜽂𜼄𜼤𜽘 𜾆𜽂𜼱𜼇 𜾒𜽃𜼱𜼈𜼤 𜽗𜽂𜼇 𜽐𜽂𜼱𜼢 𜽖𜼆 𜽗𜽃𜼅 𜽞 𜾒𜼰𜼺𜼆 𜽝𜽂𜼄 𜽜𜽃𜼅 𜾆𜽂𜼄 𜽗𜽃𜼅 𜽞 𜽐𜽂𜼱𜼇 𜽜𜼆 𜾆𜽃𜼅 𜽞 𜽐𜽃𜼱𜼈 𜽗𜽂𜼇 𜽞 𜽐𜽂𜼱𜼇 𜽖𜼆 𜽗𜽃𜼅 𜽐𜼱𜼆 𜽖𜽃𜼅 𜽜𜽂𜼄𜼦 𜾖𜼆𜼤 𜾋𜽃𜼅 𜾁
\\
Хри-сто́съ ра-жда́-ет-сѧ, сла́-ви-те: Хри-сто́съ съ нб҃съ, срѧ́-щи-те: Хри-сто́съ на зе-млѝ, воз-но-си́-те-сѧ. по́й-те Го́-спо-де-ви всѧ̀-~ зе-млѧ̀, и҆ ве-се́-лї-емъ вос-по́--йте лю́-дї-е, ꙗ҆́-кѡ про-сла́-ви-сѧ.
}
\end{churchslavonic}
\caption{Example of Type~A Znamenny Notation \label{mezenets}}
%\end{adjustbox}
\end{figure}

\subsection{Font Features}

The font provides the following stylistic sets:

\begin{itemize}
\item \textbf{ss01} (Stylistic Set 1) -- Old-style Demestvenny. This stylistic set turns
on kerning rules that force adjoining neumes to make ligatures, as is customary
in Demestvenny Notation. At the same time, the Zanozhek, Mechik, and other neumes 
are typeset in the old style, common to the manuscript tradition.

\begin{center}
\begin{tabular}{lr}
Regular         & {\musicFont 𜽐𜼆𜽖𜼢 𜾩𜼾𜼆𜽝 𜾩𜼈𜾫𜼓} \\
\textbf{ss01}   & {\oldstyle 𜽐𜼆𜽖𜼢 𜾩𜼾𜼆𜽝 𜾩𜼈𜾫𜼓} \\
\end{tabular}
\end{center}

\item \textbf{ss02} (Stylistic Set 2) -- New-style Demestvenny. This stylistic set turns
on kerning rules that force adjoining neumes to make ligatures, as is customary
in Demestvenny Notation. The Zanozhek, Mechik, and other neumes are typeset in the
new style, as used in the Demestvenny Azbuka of L.~Kalashnikov.

\begin{center}
\begin{tabular}{lr}
Regular         & {\musicFont 𜽐𜼆𜽖𜼢 𜾩𜼾𜼆𜽝 𜾩𜼈𜾫𜼓} \\
\textbf{ss02}   & {\newstyle 𜽐𜼆𜽖𜼢 𜾩𜼾𜼆𜽝 𜾩𜼈𜾫𜼓} \\
\end{tabular}
\end{center}

\item \textbf{ss03} (Stylistic Set 3) -- New-style Contracted Demestvenny. This stylistic
set turns on additional kerning rules that force some adjoining neumes to make
contracted ligatures. The Zanozhek, Mechik, and other neumes are typeset in the
new style, as used in the Demestvenny Azbuka of L.~Kalashnikov.

\begin{center}
\begin{tabular}{lr}
Regular         & {\musicFont 𜽐𜼆𜽖𜼢 𜾩𜼾𜼆𜽝 𜾩𜼈𜾫𜼓} \\
\textbf{ss03}   & {\contracted 𜽐𜼆𜽖𜼢 𜾩𜼾𜼆𜽝 𜾩𜼈𜾫𜼓} \\
\end{tabular}
\end{center}

\item \textbf{ss04} (Stylistic Set 4) -- Render cinnabar marks in red. This stylistic
set turns off the coloring of cinnabar marks specified by the COLR and CPAL tables and
renders them in black. \cuKinovar{NB:} turning on this feature will have a visible
effect only in software that supports color fonts.

\begin{center}
\begin{tabular}{lr}
Regular             & {\musicFont 𜽐𜼆𜽖𜼢 𜾩𜼾𜼆𜽝 𜾩𜼈𜾫𜼓} \\
\textbf{ss04}       & {\musicFont 𜽐𜽖 𜾩𜼾𜽝 𜾩𜾫} \\
\end{tabular}
\end{center}

\end{itemize}

In \XeTeX{} and \LuaTeX{}, features may be turned on using methods provided in the
\pkg{fontspec} package, for example:

\begin{verbatim}
\newfontfamily{\znamFont}
      [StylisticSet=1]{Mezenets Unicode}
\end{verbatim}

\noindent Please consult the \pkg{fontspec} documentation for more information.

In LibreOffice, Stylistic Sets and other advanced OpenType features are available
starting with version 4.1. However, there is no graphical interface that can be used
to turn these features on and off. Instead, a special extended font name
syntax has been developed: in order to activate an optional feature, its ID,
followed by (optionally) an equals sign and the ID of the desired setting, are appended
directly to the font name string. An ampersand is used to separate
different feature/settings pairs. For example, setting the font to the following
will enable Stylistic Set~1:

\begin{verbatim}
Mezenets Unicode:ss01
\end{verbatim}

\phantomsection
\addcontentsline{toc}{section}{Codecharts}
\includepdf[pages=-]{charts.pdf}

\end{document}