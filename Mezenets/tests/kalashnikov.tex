\documentclass[12pt]{article}
\usepackage{fontspec,luacolor}
\usepackage{longtable}
\defaultfontfeatures{Ligatures=TeX}
%\setromanfont{Linux Libertine O}
\newfontfamily{\znam}[Path=../,Extension=.otf,Scale=MatchUppercase]{MezenetsUnicode}
\definecolor{kinovar}{rgb}{0.80,0.03,0.01}
\newcommand{\cuKinovar}[1]{\textcolor{kinovar}{#1}}

\catcode`𜼀\active\def𜼀{\cuKinovar{\detokenize{𜼀}}}%
\catcode`𜼁\active\def𜼁{\cuKinovar{\detokenize{𜼁}}}%
\catcode`𜼂\active\def𜼂{\cuKinovar{\detokenize{𜼂}}}%
\catcode`𜼃\active\def𜼃{\cuKinovar{\detokenize{𜼃}}}%
\catcode`𜼄\active\def𜼄{\cuKinovar{\detokenize{𜼄}}}%
\catcode`𜼅\active\def𜼅{\cuKinovar{\detokenize{𜼅}}}%
\catcode`𜼆\active\def𜼆{\cuKinovar{\detokenize{𜼆}}}%
\catcode`𜼇\active\def𜼇{\cuKinovar{\detokenize{𜼇}}}%
\catcode`𜼈\active\def𜼈{\cuKinovar{\detokenize{𜼈}}}%
\catcode`𜼉\active\def𜼉{\cuKinovar{\detokenize{𜼉}}}%
\catcode`𜼊\active\def𜼊{\cuKinovar{\detokenize{𜼊}}}%
\catcode`𜼋\active\def𜼋{\cuKinovar{\detokenize{𜼋}}}%
\catcode`𜼌\active\def𜼌{\cuKinovar{\detokenize{𜼌}}}%
\catcode`𜼍\active\def𜼍{\cuKinovar{\detokenize{𜼍}}}%
\catcode`𜼎\active\def𜼎{\cuKinovar{\detokenize{𜼎}}}%
\catcode`𜼐\active\def𜼐{\cuKinovar{\detokenize{𜼐}}}%
\catcode`𜼑\active\def𜼑{\cuKinovar{\detokenize{𜼑}}}%
\catcode`𜼒\active\def𜼒{\cuKinovar{\detokenize{𜼒}}}%
\catcode`𜼓\active\def𜼓{\cuKinovar{\detokenize{𜼓}}}%
\catcode`𜼔\active\def𜼔{\cuKinovar{\detokenize{𜼔}}}%
\catcode`𜼕\active\def𜼕{\cuKinovar{\detokenize{𜼕}}}%
\catcode`𜼖\active\def𜼖{\cuKinovar{\detokenize{𜼖}}}%
\catcode`𜼗\active\def𜼗{\cuKinovar{\detokenize{𜼗}}}%
\catcode`𜼘\active\def𜼘{\cuKinovar{\detokenize{𜼘}}}%
\catcode`𜼙\active\def𜼙{\cuKinovar{\detokenize{𜼙}}}%
\catcode`𜼚\active\def𜼚{\cuKinovar{\detokenize{𜼚}}}%
\catcode`𜼛\active\def𜼛{\cuKinovar{\detokenize{𜼛}}}%
\catcode`𜼜\active\def𜼜{\cuKinovar{\detokenize{𜼜}}}%
\catcode`𜼝\active\def𜼝{\cuKinovar{\detokenize{𜼝}}}%
\catcode`𜼞\active\def𜼞{\cuKinovar{\detokenize{𜼞}}}%
\catcode`𜼟\active\def𜼟{\cuKinovar{\detokenize{𜼟}}}%
\catcode`𜼠\active\def𜼠{\cuKinovar{\detokenize{𜼠}}}%
\catcode`𜼡\active\def𜼡{\cuKinovar{\detokenize{𜼡}}}%
\catcode`𜼢\active\def𜼢{\cuKinovar{\detokenize{𜼢}}}%
\catcode`𜼣\active\def𜼣{\cuKinovar{\detokenize{𜼣}}}%
\catcode`𜼤\active\def𜼤{\cuKinovar{\detokenize{𜼤}}}%
\catcode`𜼥\active\def𜼥{\cuKinovar{\detokenize{𜼥}}}%
\catcode`𜼦\active\def𜼦{\cuKinovar{\detokenize{𜼦}}}%
\catcode`𜼧\active\def𜼧{\cuKinovar{\detokenize{𜼧}}}%
\catcode`𜼨\active\def𜼨{\cuKinovar{\detokenize{𜼨}}}%
\catcode`𜼩\active\def𜼩{\cuKinovar{\detokenize{𜼩}}}%
\catcode`𜼪\active\def𜼪{\cuKinovar{\detokenize{𜼪}}}%
\catcode`𜼫\active\def𜼫{\cuKinovar{\detokenize{𜼫}}}%
\catcode`𜼬\active\def𜼬{\cuKinovar{\detokenize{𜼬}}}%
\catcode`𜼭\active\def𜼭{\cuKinovar{\detokenize{}}}%
\catcode`\active\def{\cuKinovar{\detokenize{}}}%

\begin{document}

\begin{longtable}{ccp{3in}l}
\hline
No. & Neume & Comments & Variants\\
\hline
1 & \znam \Large 𜽐 &  & \\
2 & \znam \Large 𜽐𜼴 &  & \\
3 & \znam \Large 𜽐𜼳 &  & \\
4 & \znam \Large 𜽑𜼱𜼧𜼇 & (See note on color change above.) & \\
5 & \znam \Large 𜽐𜼱𜼵𜼆 &  & \\
6 & \znam \Large 𜽐𜼱𜼦𜼆 &  & \\
7 & \znam \Large 𜽒𜼆 &  & \\
8 & \znam \Large 𜽒𜼴𜼆 &  & \\
9 & \znam \Large 𜽒𜼳𜼆 &  & \\
10 & \znam \Large 𜽒𜼵𜼆 &  & \\
11 & \znam \Large 𜽒𜼦𜼆 &  & \\
12 & \znam \Large 𜽝𜼃 &  & \\
13 & \znam \Large 𜽝𜼴𜼃 &  & \\
14 & \znam \Large 𜽝𜾂𜼃 & Compose as [{\znam 𜽝 + 𜾂 = 𜽝𜾂}] with GSUB. & \\
15 & \znam \Large 𜽖𜼆 &  & \\
16 & \znam \Large 𜽖𜼴𜼆 &  & \\
17 & \znam \Large 𜽗𜼆 &  & \\
18 & \znam \Large 𜽔𜼆 &  & \\
19 & \znam \Large 𜽛𜼆 &  & \\
20 & \znam \Large 𜽱𜼱𜼆 &  & \\
21 & \znam \Large 𜽜𜼆 &  & \\
22 & \znam \Large 𜽜𜼴𜼆 &  & \\
23 & \znam \Large 𜽜𜼦𜼆 & Don’t use GSUB with this. Position the Podvertka via anchor points. & \\
24 & \znam \Large 𜽜𜼼 & Compose as [{\znam 𜽜 + ◌𜼼 = 𜾂}] with GSUB. & \\
25 & \znam \Large 𜽜𜼦𜼼𜼆𜼩 & Don’t include the Podvertka in the GSUB. Position the Podvertka with anchor points. & \\
26 & \znam \Large 𜽲 𜽳𜼆 𜽴 &  & \\
27 & \znam \Large 𜽵𜼆 &  & \\
28 & \znam \Large 𜽳𜼤 &  & \\
29 & \znam \Large 𜽳𜼣 &  & \\
30 & \znam \Large 𜽳𜼳𜼣 &  & \\
31 & \znam \Large 𜽭𜼈𜼤 &  & \\
32 & \znam \Large 𜽘𜼤 &  & \\
33 & \znam \Large 𜽘𜼧𜼅 & A suggestion for \#33, 34 and 39, if there are display problems: Type the Lomka first and then the pitches can be added via mark-to-mark positioning. & \\
34 & \znam \Large 𜽘𜼦𜼧 & (The Podvertka is red in Type B notation, and black in Type A notation.) Don’t use GSUB with this. & \\
35 & \znam \Large 𜽘𜼅𜼥 &  & \\
36 & \znam \Large 𜽞 &  & \\
37 & \znam \Large 𜽟𜼣 &  & 𜽠\\
38 & \znam \Large 𜽯𜼆 &  & \\
39 & \znam \Large 𜽯𜼧𜼅 &  & \\
(39a) & \znam \Large 𜽯𜼫 &  & \\
40 & \znam \Large 𜽰𜼅 & Should this be composed (enable anchor points {\znam 𜽯𜼻 }), or use GSUB to map to: 𜽰 ? & \\
41 & \znam \Large 𜾈𜼃 &  & \\
42 & \znam \Large 𜾉𜼃 & Compose as [{\znam 𜾈 + 𜽝 = 𜾉}] with GSUB. (?) & \znam 𜾋 \\
43 & \znam \Large 𜾉𜼣𜼅 &  & \\
44 & \znam \Large 𜾊𜼄 & Compose as [{\znam 𜾈 + 𜾂 = 𜾊}] with GSUB. (?) & \\
45 & \znam \Large 𜾒𜼣𜼅 &  & \\
46 & \znam \Large 𜾈𜼰𜼅 & \znam (𜾈 + ◌𜼰 ) &  \\
47 & \znam \Large 𜾔𜼆 &  & 𜾜\\
48 & \znam \Large 𜾔𜼰𜼆𜼢 &  & \\
49 & \znam \Large 𜾔𜼰𜼺𜼅𜼢𜼪 &  & \\
50 & \znam \Large 𜾔𜼵𜼆 &  & \\
51 & \znam \Large 𜾔𜼦𜼇𜼤 &  & \\
52 & \znam \Large 𜾈𜼱𜼆 & \znam (𜾈 + ◌𜼱 ) & \\
53 & \znam \Large 𜾈𜼱𜼣𜼈 &  & \\
54 & \znam \Large 𜾈𜼱𜼹𜼊𜼢 & \znam (𜾈 + ◌𜼱 + ◌𜼹 ) & \\
55 & \znam \Large 𜾈𜼱𜼺𜼉 & \znam (𜾈 + ◌𜼱 + ◌𜼺 ) & \\
56 & \znam \Large 𜾍𜼆𜼢 &  & 𜾎\\
57 & \znam \Large 𜾍𜼰𜼈𜼢 & {\znam (𜾍 + ◌𜼰 ) } GSUB works correctly in LibreOffice Calc, but it doesn’t work in LibreOffice Writer. & \\
58 & \znam \Large 𜾍𜼰𜼄𜼣𜼢 & {\znam (𜾍 + ◌𜼰 ) } Ditto & \\
59 & \znam \Large 𜾂 &  & \\
60 & \znam \Large 𜾃 &  & \\
61 & \znam \Large 𜽾𜼆 &  &  \\
62 & \znam \Large 𜽾𜼆𜼨 &  &  \\
63 & \znam \Large 𜽾𜼳𜼆 & \znam (𜽾 + ◌𜼳 ) &  \\
64 & \znam \Large 𜽾𜼦𜼆 & {\znam (𜽾 + ◌𜼦 )} Add GSUB to  (?). Keep in mind that this can be composed in a number of different ways (including a variant with the Podvertka in red ink), especially in Type B notation, and it will be important to find ways to make this possible. This is a work item for later. &  \\
65 & \znam \Large 𜽾𜼦𜼆𜼨 & {\znam (𜽾 + ◌𜼦 )} Ditto &  \\
66 & \znam \Large 𜼆 & Should we use GSUB using {\znam 𜽾 + 𜽝 } to point to U+EAF1? Or will this foul things up for adding priznaki? &   \\
67 & \znam \Large 𜼇𜼣 & Ditto & \znam  \\
68 & \znam \Large 𜼦𜼇 & TODO: Remove GSUB to {\znam 𜼦  } because we need to have color separation. &  \\
69 & \znam \Large 𜾒𜼰𜼇 & {\znam (𜾒 + ◌𜼰 )} GSUB seems to be working correctly in LibreOffice Calc and Writer. & \\
70 & \znam \Large 𜽓𜼆𜼩 &  & \\
71 & \znam \Large 𜽓𜼳𜼆𜼩 &  & \\
72 & \znam \Large 𜽓𜼴𜼆𜼩 &  & \\
73 & \znam \Large 𜽓𜼦𜼆𜼩 &  & \\
74 & \znam \Large 𜽓𜼵𜼆𜼩 &  & \\
75 & \znam \Large 𜿃𜼇𜼣𜼤 & (Should this be composed?) & \\
76 & \znam \Large 𜿃𜼰𜼹𜼇𜼣𜼤𜼢 & \znam (𜿃 + ◌𜼰 + ◌𜼹 ) & \\
77 & \znam \Large 𜿂𜾉𜼆 & (composed) Fix position of pitch mark. & \\
78 & \znam \Large 𜾰𜼇 &  & \\
79 & \znam \Large 𜿄𜾉𜼇 & (composed) Fix position of pitch mark. & \\
80 & \znam \Large 𜾩𜼆 &  & \\
81 & \znam \Large 𜾕𜼰𜼣𜼆 &  & \\
82 & \znam \Large 𜾕𜼳𜼰𜼣𜼆 &  & \\
83 & \znam \Large 𜾕𜼰𜼹𜼆𜼣 & \znam (𜾕 + ◌𜼰 + ◌𜼹 ) & \\
84 & \znam \Large 𜾖𜼰𜼆𜼣 &  & \\
85 & \znam \Large 𜾖𜼰𜼳𜼆𜼣 &  & \\
86 & \znam \Large 𜾖𜼰𜼆𜼤 &  & \\
87 & \znam \Large 𜾖𜼰𜼳𜼆𜼤 &  & \\
88 & \znam \Large 𜾖𜼰𜼺𜼇𜼤𜼪 &  & \\
89 & \znam \Large 𜾖𜼰𜼹𜼇𜼣 & \znam (𜾖 + ◌𜼰 + ◌𜼹 ) & \\
90 & \znam \Large 𜾖𜼰𜼹𜼇𜼤 & Ditto & \\
91 & \znam \Large 𜾖𜼰𜼹𜼆𜼣𜼢 & Ditto & \\
92 & \znam \Large 𜾖𜼰𜼹𜼆𜼤𜼢 & Ditto & \\
93 & \znam \Large 𜾘𜼆𜼤 &  & \znam 𜾗 𜾙\\
93a & \znam \Large 𜾘𜼅𜼥 &  & \\
94 & \znam \Large 𜾘𜼆𜼣 &  & \\
95 & \znam \Large 𜾘𜼳𜼆𜼤 &  & \\
96 & \znam \Large 𜾘𜼦𜼆𜼤 &  & \\
97 & \znam \Large 𜾔𜼱𜼇𜼤 &  & \\
98 & \znam \Large 𜾔𜼳𜼱𜼇𜼤 &  & \\
99 & \znam \Large 𜾔𜼱𜼺𜼇𜼤 &  & \\
100 & \znam \Large 𜾔𜼱𜼦𜼇𜼤 &  & \\
101 & \znam \Large 𜾔𜼱𜼵𜼇𜼤 &  & \\
102 & \znam \Large 𜾔𜼱𜼇𜼣 &  & \\
103 & \znam \Large 𜾔𜼳𜼱𜼇𜼣 &  & \\
104 & \znam \Large 𜾔𜼱𜼵𜼇𜼣 &  & \\
105 & \znam \Large 𜾕𜼱𜼇𜼣 &  & \\
106 & \znam \Large 𜾕𜼱𜼳𜼇𜼣 &  & \\
107 & \znam \Large 𜾘𜼰𜼇𜼤 &  & \\
108 & \znam \Large 𜾤𜼁 &  & \\
109 & \znam \Large 𜾤𜼄𜼤 &  & \\
110 & \znam \Large 𜾤𜼆𜼩 &  & \\
111 & \znam \Large 𜾤𜼳𜼆𜼩 &  & \\
112 & \znam \Large 𜾤𜼦𜼆𜼩 &  & \\
113 & \znam \Large 𜾤𜼵𜼆𜼩 &  & \\
114 & \znam \Large 𜾤𜼺𜼆𜼩𜼪 &  & \\
115 & \znam \Large 𜾤𜼱𜼈𜼤 &  & \\
116 & \znam \Large 𜾥𜼳𜼆𜼤 &  & \\
117 & \znam \Large 𜾤𜼱𜼦𜼈𜼤 &  & \\
118 & \znam \Large 𜾤𜼱𜼵𜼈𜼤 &  & \\
119 & \znam \Large 𜾤𜼱𜼺𜼇𜼤 &  & \\
120 & \znam \Large 𜾥𜼆𜼣 &  & \\
121 & \znam \Large 𜾤𜼱𜼳𜼈 &  & \\
122 & \znam \Large 𜾧𜼱𜼈𜼤 &  & \znam \Large 𜾨\\
123 & \znam \Large 𜾧𜼰𜼳𜼅𜼤 &  & \\
124 & \znam \Large 𜾧𜼱𜼦𜼆𜼤 &  & \\
125 & \znam \Large 𜾧𜼲𜼵𜼉𜼤 &  & \\
126 & \znam \Large 𜾧𜼱𜼺𜼈𜼤 &  & \\
127 & \znam \Large 𜾧𜼱𜼹𜼈𜼤 & \znam \Large (𜾧 + ◌𜼱 + ◌𜼹 ) & \\
128 & \znam \Large 𜾧𜼱𜼹𜼆𜼤𜼢 & Ditto & \\
129 & \znam \Large 𜾭𜼈𜼥 &  & \\
130 & \znam \Large 𜾭𜼦𜼈𜼥 &  & \\
131 & \znam \Large 𜾭𜼵𜼈𜼥 &  & \\
132 & \znam \Large 𜾯𜼰𜼳𜼆𜼥 &  & \\
133 & \znam \Large 𜾮𜼰𜼈𜼥 &  & \\
134 & \znam \Large 𜿅 &  & \\
135 & \znam \Large 𜿆𜼅𜼤 &  & \\
136 & \znam \Large 𜾍𜼃𜼢 𜿆𜼅𜼤 &  & \\
137 & \znam \Large 𜽮 &  & \\
138 & \znam \Large 𜼴𜼇𜼥 &  & \\
139 & \znam \Large 𜾄 &  & \\
140 & \znam \Large 𜾒 &  & \\
141 & \znam \Large 𜼢𜼣 & {\znam (𜾍 + ◌𜼰 + ◌𜼦 )} TODO: Add GSUB to invoke {\znam  }. (This should be composed at the keyboard entry level, invoking GSUB, particularly because the Podvertka is sometimes red in Type B notation. In Type A notation the Podvertka is always black, but it is probably best to add it via anchor positioning, rather than a full GSUB that includes the Podvertka. We want our end users to have the option to add the Podvertka as red or black.) & \znam \\
142 & \znam \Large 𜾇 &  & \znam \Large 𜾆  \\
\hline
\end{longtable}
\end{document}